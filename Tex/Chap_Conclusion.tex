\chapter{结束语}\label{chap:Conclusion}

\section{本文工作总结}
本文研究对AMD GPU矩阵乘算法进行并行加速。 

本文首先分析了AMD GPU架构细节,分析了当前主流GPU矩阵乘的优化工作。总结了NVIDIA GPU矩阵乘的优化手法,将该手法结合AMD Fiji和Vega架构,设计了Fiji和Vega架构矩阵乘,并进行了调优工作。实现了AMD GPU高效的矩阵乘。

在开始做AMD GPU矩阵乘调优时,使用了AMD Fiji和Vega GPU。首先通过AMD官网找到Fiji和Vega GPU架构的ISA手册,阅读与分析ISA手册,开始对AMD GPU的架构有所了解。接下来开始测AMD的rocBLAS,MIOpenGEMM和Tensile等数学库的矩阵乘性能,并分析其中的GEMM kernerl。一开始的研究计划是同时考虑做自动调优和手工调优。在现有的AMD GPU开源数学中,有官方的rocBLAS库,也有第三方开源库如CLBlast\citepns{nugteren2017clblast}。CLBlast目前维护的比较好,其整体性能要略好于rocBLAS,是一个很好的工作。AMD GPU现有GEMM自动调优库有MIOpenGEMM和Tensile。从性能上,Tensile要好于MIOpenGEMM,Tensile可以将矩阵乘的性能通过自动调优接近91\%。缺点是Tensile生成kernel的时间要比MIOpenGEMM长很多。MIOpenGEMM目前集成在MIOpen卷积数学库中。Tensile集成在rocBLAS中。我用了较多时间阅读MIOpenGEMM和Tensile库的源码,分析其中的技术要点和设计原理。手工优化GEMM的方法主要来源于Junjie Lai在Fermi和Kepler GPU上的GEMM调优工作和GEMM性能上界分析以及Scott Gray针对Maxwell架构做的maxas SGEMM的工作。

此外,本文分析了AMD GPU矩阵乘的性能上界,可以知道我们当前实现的矩阵乘效率距离矩阵乘性能上界还有多远。这种分析方法具有通用性,可以将这种方法稍加修改就可以应用于其他GPU应用程序性能上界的分析。为我们的调优工作提供了指导方向。

\section{下一步工作}
本文的工作是基于AMD GPU的通用矩阵乘调优。采用手工汇编优化的手法,通过寄存器分块,寄存器双缓冲,shared memory双缓冲等,指令调度等手法一步步提高矩阵乘性能。采用手工汇编优化的优点是可以通过十分细粒度的指令调度将矩阵乘的性能尽可能调高;同时也有缺点,一方面进行手工优化需要深入理解AMD GPU的架构细节,理解AMD GPU的内存层次结构,指令的调度方式,寄存器,shared memory等资源的大小,另一方面也要有丰富的程序优化经验,和汇编程序编制和调试的编程基础。同时,如果AMD GPU架构发生变化时,我们就要重复这种调优工作。另一种做AMD GPU矩阵乘的方向是自动调优,通过benchmark的方法自动找出较优的矩阵乘kernel。自动调优可以自动生成针对不同架构的矩阵乘kernel。免去了手工重复调优的工作。但一般来说,手工优化的kernel性能会高于自动调优工具。

自动调优会涉及到编译领域的相关技术。编写代码模板,代码自动生成等。自动调优可提高kernel生成的速度,加快实际应用的部署,是一个具有研究价值的领域。

自动调优工具在设计和实现时,其实已经加入了一定的优化经验在里面。自动调优工具可以是有一份或多份代码模板,在编写模板是,就加了优化的经验,同时外加了对性能有重要影响的关键可变参数。通过尝试不同的参数组合,来生成一组kernel,测试这些kernel的性能,选出比较优的kernel作为实际中采用的kernel。对于设计良好的自动调优工具,其生成的矩阵乘性能已经十分接近手工优化的性能,可以满足大部分应用的需求。

所以,未来做自动调优工具将是优化程序性能的一个重要方向。
