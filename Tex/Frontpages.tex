%---------------------------------------------------------------------------%
%->> 封面信息及生成
%---------------------------------------------------------------------------%
%-
%-> 中文封面信息
%-
\confidential{}% 密级:只有涉密论文才填写
\schoollogo{scale=0.095}{ucas_logo}% 校徽
\title{面向GPU体系结构的通用矩阵乘优化研究}% 论文中文题目
\author{郝海波}% 论文作者
%\advisor{马捷~正研级高工~中国科学院计算技术研究所}% 指导教师:姓名 专业技术职务 工作单位
\advisor{马捷~正研级高级工程师}% 指导教师:姓名 专业技术职务 工作单位
\advisorsec{中国科学院计算技术研究所}% 指导老师附加信息 或 第二指导老师信息
\degree{硕士}% 学位:学士、硕士、博士
\degreetype{工程}% 学位类别:理学、工学、工程、医学等
\major{计算机技术}% 二级学科专业名称
\institute{中国科学院计算技术研究所}% 院系名称
\chinesedate{2018~年~5~月}% 毕业日期:夏季为6月、冬季为12月
%-
%-> 英文封面信息
%-
\englishtitle{General Matrix Multiplication optimization for\\ GPU architecture}% 论文英文题目
\englishauthor{Hao Haibo}% 论文作者
\englishadvisor{Supervisor: Professor Ma Jie}% 指导教师
\englishdegree{Master}% 学位:Bachelor, Master, Doctor。封面格式将根据英文学位名称自动切换,请确保拼写准确无误
\englishdegreetype{Engineering}% 学位类别:Philosophy, Natural Science, Engineering, Economics, Agriculture 等
\englishthesistype{thesis}% 论文类型: thesis, dissertation
\englishmajor{Computer Technology}% 二级学科专业名称
\englishinstitute{Institute of Computing Technology\\ Chinese Academy of Sciences}% 院系名称
\englishdate{May, 2018}% 毕业日期:夏季为June、冬季为December
%-
%-> 生成封面
%-
\maketitle% 生成中文封面
\makeenglishtitle% 生成英文封面
%-
%-> 作者声明
%-
\makedeclaration% 生成声明页
%-
%-> 中文摘要
%-
\chapter*{摘\quad 要}\chaptermark{摘\quad 要}% 摘要标题
\setcounter{page}{1}% 开始页码
\pagenumbering{Roman}% 页码符号

%当前,一颗GPU芯片上集成的核心数越来越多,GPU体系结构也在非常快速的演变。目前主要的高端GPU芯片厂商有NVIDIA和AMD。主流的GPUs架构有NVIDIA Kepler,Maxwell,Pascal和Volta GPU;AMD Fiji,Vega10,Vega20 GPU。由于每一代的GPU架构都会发生变化,我们就需要在新的架构上重新做优化工作。不幸的是,我们没有可用的性能上界分析方法和工具。在实际中,开发人员通过算法的分析和积累的经验,采用多种优化手段来编写高效的kernel。Kernel编写人员可能会通过性能分析工具(如NVVP\citepns{profiler2011nvidia})的分析结果来指导进一步的优化。然而,这样并不能知道现在优化的结果距离性能上界还有多远。
%
%实现GPU (Graphic Processing Units)上的快速通用矩阵乘一直以来都是自动调优工具和kernel编写人员所追求的目标。在本篇文章中,我们尝试提供一种GPUs上算法性能上界的方法,并做出汇编级的基准测试。Meng等人\citepns{meng2011grophecy}提出了GPU性能预测框架,该框架基于标注代码框架。Hong和Kim提出了MWP-CWP\citepns{hong2009analytical}模型来预测CUDA应用程序性能,该模型基于NVIDIA PTX。Sim\citepns{sim2012performance}等人在2012年将MWP-CWP模型进行了扩展,使用汇编编写kernel预测程序性能。Zhang和Owen\citepns{zhang2011quantitative}基于汇编程序提出了GPU定量分析模型。由于关于GPU微架构的资料非常少,我们无法针对新一代GPU架构做出精确的GPU模拟器。但我们通过上面这些分析方法可以十分近似的预测GPU程序的性能。Roofline\citepns{williams2009roofline}模型是应用最广的用来评估优化效果的模型,本文将采用该模型对SGEMM做性能分析。
%
%对于AMD GPU,其官方并没有提供良好的性能分析工具。但AMD GPU有可用的LLVM汇编器。本文通过手工汇编优化的手法,在Fiji 和Vega GPU上实现的矩阵乘性能达到95\%。
当前,一颗GPU芯片上集成的核心数越来越多,GPU体系结构也在非常快速的演变。目前主要的高端GPU芯片厂商有NVIDIA和AMD。主流的GPUs架构有NVIDIA Maxwell,Pascal和Volta GPU;AMD Fiji,Vega GPU。由于每一代的GPU架构都会发生变化,编程人员就需要在新的架构上重新做优化工作。

矩阵乘是BLAS(Basic Liner Algebra Subprogram) Level3标准中定义的数学计算,是BLAS库的核心子程序之一。矩阵乘的效率通常可以展示一个计算机系统实际可达的最高计算性能。通过对矩阵乘性能的调优和性能分析,可以深入理解现代多核处理器体系结构。同时也将提高BLAS库Level3的计算性能,从而提高基于BLAS库的应用的计算速度。

本文工作基于AMD GPU平台,实现了面向GPU体系结构的通用矩阵乘优化,以解决当下AMD GPU矩阵乘的性能低的问题,并且对AMD GPU矩阵乘进行了性能上界分析,为优化AMD GPU其他应用程序提供了性能分析方法。论文的主要针对新一代AMD GPU体系结构的特点,设计出一套AMD GPU微基准测试程序,并探测出AMD GPU浮点指令的通量和内存带宽;实现了面向AMD GPU体系结构的通用矩阵乘计算,通过对AMD GPU架构的理解,利用指令预取、双缓冲、bank冲突消除、指令重排等调优方法,提高了AMD GPU矩阵乘的效率;通过对矩阵乘主循环中浮点指令和访存指令在不同百分比下指令通量的测试,分析了矩阵乘的性能上界,并将Roofline模型应用到AMD GPU矩阵乘的性能分析中。


\keywords{矩阵乘,AMD GPU,汇编,延迟掩盖,Roofline模型}% 中文关键词
%-
%-> 英文摘要
%-
\chapter*{Abstract}\chaptermark{Abstract}% 摘要标题

Currently, the number of cores integrated on a GPU chip is increasing, and the GPU architecture is also rapidly evolving. At present, the main high-end GPU chip manufacturers are NVIDIA and AMD. The mainstream GPUs are NVIDIA Maxwell, Pascal and Volta GPUs, AMD Fiji, and Vega GPUs. As each generation of GPU architecture changes, programmers need to redo the optimization work on the new architecture.

Matrix multiplication is a mathematical calculation defined in the Basic Liner Algebra Subprogram Level 3 standard and is one of the core subroutines of the BLAS library. The efficiency of matrix multiplication usually shows the highest computational performance that a computer system can actually achieve. Through the optimization and performance analysis of matrix multiplication performance, a deep understanding of modern multi-core processor architecture can be obtained. At the same time, the computing performance of Level 3 of the BLAS library will also be improved, thereby increasing the calculation speed of applications based on the BLAS library.

This paper is based on the AMD GPU platform and implements general matrix multiplication optimization for GPU architecture to solve the problem of low performance of current AMD GPU matrix multiplication, and performs performance bound analysis on AMD GPU matrix multiplication to optimize AMD GPU other. The application provides a performance analysis method. The paper focuses on the characteristics of the next-generation AMD GPU architecture, designing a set of AMD GPU micro benchmarks, detecting the throughput and memory bandwidth of AMD GPU floating-point instructions, and achieving universal matrix multiplication for AMD GPU architectures. Calculation, through the understanding of AMD GPU architecture, using instruction prefetching, double buffering, bank collision elimination, instruction rearrangement and other tuning methods to improve the efficiency of AMD GPU matrix multiplication; through the matrix multiplication of floating point instructions in the main loop and The memory access instruction was tested at different percentages of the instruction throughput, and the upper bound of matrix multiplication performance was analyzed. The Roofline model was applied to the performance analysis of AMD GPU matrix multiplication.


\englishkeywords{GEMM, AMD GPU, Assembly, Latency hiding, Roofline model}% 英文关键词
%---------------------------------------------------------------------------%
