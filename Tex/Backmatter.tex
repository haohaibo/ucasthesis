\chapter[致谢]{致\quad 谢}\chaptermark{致\quad 谢}% syntax: \chapter[目录]{标题}\chaptermark{页眉}
\thispagestyle{noheaderstyle}% 如果需要移除当前页的页眉
%\pagestyle{noheaderstyle}% 如果需要移除整章的页眉

在我的硕士课题和论文即将完成之际,谨在此向攻读硕士学位期间关心和指导我的老师,以及一直以来支持和帮助我的家人、朋友和同学致以崇高的敬意和衷心的感谢!

首先,感谢我的导师马捷研究员和谭光明研究员!马老师平日工作十分繁忙,却从未因此忽视对学生的指导与关心。我更要感谢谭老师,在我整个研究生期间对我的研究工作的殷切指导。从选择课题到完成论文,谭老师始终都是耐心地指导,并给予不懈的支持。马老师和谭老师敏锐的洞察力、渊博的知识、严谨的治学态度、精益求精的敬业精神,无论我以后从事何种职业,都值得我学习!

感谢张春明老师、姚二林老师,有你们这样的良师益友,在我学习、工作、生活上给予关心、帮助,让我受益良多。感谢邵老师,作为我们的辅导员,在生活上处处为我们着想,解决我们遇到的问题。感谢秘书处的何玉晓和龙文艳老师,是你们细致入微的工作让我们能够安心进行科研工作。

感谢实验室的员工李旭师兄、张中海师兄,感谢李雪琦师兄、王元戎师兄、王炳琛师兄、曾平师兄和张秀霞,刘军红师姐,你们在科研工作和生活学习等方面都给予我很多宝贵的建议,让我少走很多弯路。感谢宋佳、曹旭、潘玉路、王彪、张云尧同学,能与你们一起度过宝贵的研究生时光,给了我很多难忘的回忆。感谢于献智、张晓扬师弟和高艳珍师妹,你们的活力和认真的态度无时无刻不在感染着我。

最后,我想感谢我的父母和家人对于我这么多年来的支持,你们让我更加乐观地面对生活中的一切,你们对我如此无私的付出,我会用一生去回报。感谢我的朋友,是你们在我承受巨大压力时帮我舒缓,在我遇到困难时伸出援手。感谢我生命中的每一个人!



\rightline{郝海波}
\rightline{2018年5月}

\cleardoublepage[plain]% 让文档总是结束于偶数页,可根据需要设定页眉页脚样式,如 [noheaderstyle]


%\chapter{作者简历及攻读学位期间发表的学术论文与研究成果}
\chapter{作者简介}
%\textbf{本科生无需此部分}。

%\section*{作者简历}
%\section*{作者简历}

%\subsection*{casthesis作者}

%吴凌云,福建省屏南县人,中国科学院数学与系统科学研究院博士研究生。
%姓名:郝海波\qquad 性别:男\qquad 出生日期:1991.08.22\qquad 籍贯:陕西渭南\\
%
%2015.9 – 2018.6\qquad\qquad 中国科学院计算技术研究所研究生\\
%2011.9 – 2015.6\qquad\qquad 西安电子科技大学本科生
\begin{flushleft}
	姓名:郝海波\qquad 性别:男\qquad 出生日期:1991.08.22\qquad 籍贯:陕西渭南\\
	\vspace{2ex}
	2015.9 – 2018.6\qquad\qquad 中国科学院计算技术研究所研究生\\
	2011.9 – 2015.6\qquad\qquad 西安电子科技大学本科生
\end{flushleft}
%2015.9 – 2018.6\qquad\qquad 中国科学院计算技术研究所研究生\\
%2011.9 – 2015.6\qquad\qquad 西安电子科技大学本科生

%\subsection*{ucasthesis作者}
%
%莫晃锐,湖南省湘潭县人,中国科学院力学研究所硕士研究生。
%
%\section*{已发表(或正式接受)的学术论文:}
%
%[1] ucasthesis: A LaTeX Thesis Template for the University of Chinese Academy of Sciences, 2014.
%
\section*{申请或已获得的专利:}

一种可利用AMD GPU汇编指令加速的单精度矩阵乘优化方法
%
%(无专利时此项不必列出)
%
\section*{攻读硕士学位期间参加的科研项目:}
E 级计算机关键技术验证系统


\section*{攻读硕士学位期间获奖荣誉情况:}
2018年中国科学院大学三好学生
%
%可以随意添加新的条目或是结构。

%\chapter[致谢]{致\quad 谢}\chaptermark{致\quad 谢}% syntax: \chapter[目录]{标题}\chaptermark{页眉}
%\thispagestyle{noheaderstyle}% 如果需要移除当前页的页眉
%%\pagestyle{noheaderstyle}% 如果需要移除整章的页眉
%
%感激casthesis作者吴凌云学长,gbt7714-bibtex-style
%开发者zepinglee,和ctex众多开发者们。若没有他们的辛勤付出和非凡工作,\LaTeX{}菜鸟的我是无法完成此国科大学位论文\LaTeX{}模板ucasthesis的。在\LaTeX{}中的一点一滴的成长源于开源社区的众多优秀资料和教程,在此对所有\LaTeX{}社区的贡献者表示感谢!
%
%ucasthesis国科大学位论文\LaTeX{}模板的最终成型离不开以霍明虹老师和丁云云老师为代表的国科大学位办公室老师们制定的官方指导文件和众多ucasthesis用户的热心测试和耐心反馈,在此对他们的认真付出表示感谢。特别对国科大的赵永明同学的众多有效反馈意见和建议表示感谢,对国科大本科部的陆晴老师和本科部学位办的丁云云老师的细致审核和建议表示感谢。谢谢大家的共同努力和支持,让ucasthesis为国科大学子使用\LaTeX{}撰写学位论文提供便利和高效这一目标成为可能。
%
%\cleardoublepage[plain]% 让文档总是结束于偶数页,可根据需要设定页眉页脚样式,如 [noheaderstyle]

